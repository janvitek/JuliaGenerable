\documentclass{paper}
\usepackage{fullpage}
\usepackage{style/julia}    %% Julia code
\usepackage{xspace}
\usepackage{mathtools}      %% for \prescript
\usepackage{fdsymbol}       %% \lessclosed
\usepackage{mathpartir}     %% inference rules
\usepackage[linesnumbered]{algorithm2e}  %% algorithms
\usepackage{threeparttable}
\usepackage[autolanguage]{numprint}
\usepackage{stmaryrd}
\usepackage{multirow}

\renewcommand{\c}{\cjl}
\newcommand{\cjl}[1]{{\small\lstinline[language=Julia]!#1!}\xspace}

%% *********************************************************
\newcommand{\EM}[1]{\ensuremath{#1}\xspace}    %% make sure we are in math mode
\newcommand{\Gen}[2]{\EM{#1\,\rightsquigarrow\,#2}}  %% Generate subtypes
\newcommand{\Set}[1]{\EM{\{ #1\}}}                   %% A set of things
\newcommand{\Rule}[2]{\inferrule[]{#1}{#2}}          %% An inference rule
\newcommand{\RuleL}[3]{
  \inferrule[\textcolor{gray}{\footnotesize [#3\!]}]{#1}{#2}
} 
\newcommand{\Exist}[2]{\EM{\exists #1\!.\,#2}}       %% A #1 exists in #2
\newcommand{\Tuple}[1]{\EM{\langle #1 \rangle}}      %% A tuple of #1
\newcommand{\Union}[1]{\EM{\cup(#1)}}                %% A union of #1
\newcommand{\m}[1]{\EM{{#1^{*}}}}                     %% Zero or more
\renewcommand{\t}{\EM{t}}                            %% A type t
\newcommand{\mt}{\EM{\m{\t}}}                        %% Zero or more t s
\newcommand{\X}{\EM{X}}                              %% Variable X
\newcommand{\Y}{\EM{Y}}                              %% Variable Y
\newcommand{\Xt}{\EM{\X^\t}}                         %% Variable X bounded by t
\newcommand{\Ty}[2]{\EM{#1(#2)}}                     %% Datatype #1(#2)
\newcommand{\T}{\EM{T}}                              %% Datatype name T
\newcommand{\TyT}[1]{\EM{\Ty \T{#1}}}                %% Datatype T(#1)
\newcommand{\TyTt}{\EM{\TyT{\mt}}}                   %% Datatype T(t*)
\newcommand{\TyTX}{\EM{\TyT{\m\X}}}                  %% Datatype T(X*)
\renewcommand{\v}{\EM{v}}                            %% Value v
\renewcommand{\d}{\EM{d}}                            %% Declaration d 
\renewcommand{\l}{\EM{l}}                            %% Left hand side l
\newcommand{\Unify}[4]{\EM{\mathit{unify}(#1,#2,#3)=#4}}
\newcommand{\Setify}[1]{\EM{[\![\,#1\,]\!]}}
\newcommand{\Subst}[3]{\EM{#1[#2\!/#3]}}             %% Subsitution
\newcommand{\Match}[4]{\EM{\mathit{match}(#1,#2,#3) = #4}}

\begin{document}

\title{Generating subtypes}
\author{A/J/J}
\maketitle
\begin{abstract}
Notes about how to generates subtypes
\end{abstract}

\section{Introduction}

This note aims to describe how to generate subtypes of Julia method signature
for the purpose of testing type stability.

The definition of Julia is imprecise, so generating the exact set of
subtypes of a signature may not be possible. We choose to generate a
superset of that set of types, and rely on the Julia type checker to
filter out extraneous types.

We strive to limit the size of the superset for perfomance reasons, the space
of types is large and checking any type is costly.

\section{A specification of the generator}

The Julia type language is rich and its semantics is complex. Previous
work has shown that the subtyping relation is not decidable.

Moreover, even the Julia language specification admits to being at best
heuristic (See the discussion of Diagonal types).

Our approach is to start with a specification for a generator that is
simple but ignores some important features of the language. As a
specification, we aim for readability rather than as a guide towards a
practical implementation.

We will then discuss extensions and implemtation issues.

\subsection{Core Julia}

\newcommand{\Any}{\EM{\mathit{Any}}}

Julia has an idiosyncratic syntax and terminology, this section
presents the type language using more traditional
terminology. Metavariables \t, \v and \d range over, respectively,
types, type instances, and type declarations.  The notation \m\t
stands for zero or more types \t and $\t_1\dots\t_n$ for the same.
The type grammar is given in Fig.~\ref{g}.  For the remainder, we
shall assume that types are well-formed (as defined in the text).

\begin{figure}[!h]
\[\begin{array}{cl@{\quad}ll}
\v   ::= & \TyTt            & \text{value} \\
         & \Tuple\mt        & \text{} \\
\t   ::= & \v               & \text{type} \\
         & \X               & \text{} \\
         & \Union\mt        & \text{} \\          
         & \Exist \Xt \t    & \text{} \\
\d ::=   & \l <: \TyTt      & \text{declaration}\\
\l ::=   & \TyTX            & \text{} \\
         & \Exist \Xt\l     & \\
 \end{array}\]
\caption{Grammar}\label{g}
\end{figure}

A type declaration \d such as $\Exist\Xt{\TyT\X} <: \Ty{\T'}\X$
introduces a type constructor \TyT\X parameterized by an existentially
quantified type variable \X upper bounded by type \t. Furthermore,
this type is declared to be a subtype of \Ty{\T'}\X. The scope of the
type variable extends to the parent type. Type constructors without
parameters can be written without parentheses, i.e. $\TyT{} \equiv
\T$. The declaration of type \Ty{\Any}{} is assumed.  In general a
declarations may introduce multiple variables and provide any type to
parent. For instance,
$\Exist{\X^\Any}{\Exist{\Y^\Any}{\TyT{\X,\Y}}}<:\Ty{\T'}{\X,\Any}$
introduce the type constructor \TyT{\X,\Y} as a subtype of $\T'$, both
variables are bounded by \Any and the parent type is instantiated with
\Ty{\T'}{\X,\Any}. A type declaration is well-formed if it has no free
variable, all of its constituent types are well-formed, and the
instantiation of the parent type constructor respects that types'
bounds. In Julia terms, type declarations cover all of abstract,
primitive and composite type declarations. Concrete types are types
that have no subtypes. For our purposes, we ignore the \c{mutable}
qualifier as it does not affect subtyping ---mutable types are limtied
to concrete types--- and size modifiers for primitive types.

A type, \t, can be an instantiation of a type constructor, a tuple of
types \Tuple\mt, a variable \X, the union of types \Union\mt and a
variable introduction \Exist\Xt\TyTt.  In terms of subtyping, Julia
limits parametric types to be invariant in their arguments, so that
\c{Ref\{Int\}<:Ref\{Real\}} does not hold even if \c{Int} is a subtype
of \c{Real}. On the other hand, tuples are co-variant, and thus
\c{Tuple\{Int\}<:Tuple\{Real\}} holds. Julia supports lower bounds for
existential, but they are rarely used. Not supporting them, entails
the generation of more subtypes than necessary. Well-formed types have
correct instantiations of parametric types, no undefined type
constructor and, at the top-level, no free variables.

An instance \v is the type of a run-time value in Julia, these can be
either type constructors or tuples. Note that the inner types can be
parametric.


\subsection{A Core Generator}

\newcommand{\RSet}{{\sf GenSet}\xspace}
\newcommand{\RUnion}{{\sf GenUnion}\xspace}
\newcommand{\RTuple}{{\sf GenTuple}\xspace}
\newcommand{\RExists}{{\sf GenExists}\xspace}
\newcommand{\RDeclare}{{\sf GenDeclare}\xspace}

A generator is a relation \Gen\t S that, for a set of type
declarations \m\d, generates all subtypes of a type \t where $S$ is a
\Set{\v_1\dots\v_n} of value types.  The relation is specified in
Fig.~\ref{gen}. The rules are non-deterministic, each application of a
generator returns \emph{all} derivable types. Each of the rules is
described next.

\newcommand{\UnifyGround}[2]{\EM{#1 \cong #2}}

\begin{figure}[!h]\begin{mathpar} %%%%%%%%%%%%%%%%%%%%%%%%%%%%%%%%%%%%%%%%%%%%%
\RuleL{                                        %% Set of values %%%%%%%%%%%%%%%
  \Gen {\v_1}{S_1}~\dots~\Gen{\v_n}{S_n}
}{
  \Gen{\{\v_1\dots\v_n\}}{S_1\cup\dots\cup S_n}
}\RSet

\RuleL{                                        %% Unions %%%%%%%%%%%%%%%%%%%%%%
 \Gen {t_1}{S_1} ~\dots~ \Gen {t_n}{S_n}   
}{
  \Gen{\Union{\t_1\dots\t_n}}{S_1\cup\dots\cup S_n}
}\RUnion

\RuleL{                                        %% Tuples %%%%%%%%%%%%%%%%%%%%%%
 \Gen {t_1}{S_1} ~\dots~ \Gen{t_n}{S_n} 
}{
  \Gen{\Tuple{\t_1\dots \t_2}}{\Setify{\Tuple{S_1\dots S_n}}}
}\RTuple

\RuleL{                                        %% Exist %%%%%%%%%%%%%%%%%%%%%%%
  \Gen\t{S'} \and \Gen{\Setify{\Subst{\t'}\X{S'}}}S
}{
  \Gen{\Exist\Xt{\t'}}S
}\RExists

\RuleL{                                         %% Declaration %%%%%%%%%%%%%%%%
  \l <: \TyT{\m{\t_2}} \and
   \Match\l{\TyT{\m{\t_1}}}{\TyT{\m{\t_2}}}\t \and  \Gen\t S
}{
  \Gen {\TyT{\m{\t_1}}}{S\cup \TyT{\m{\t_1}}}
}\RDeclare
\end{mathpar} %%%%%%%%%%%%%%%%%%%%%%%%%%%%%%%%%%%%%%%%%%%%%%%%%%%%%%%%%%%%%%%%%
\caption{Generation rules}\label{gen}

\begin{mathpar} %%%%% Matching %%%%%%%%%%%%%%%%%%%%%%%%%%%%%%%%%%%%%%%%%%%%%%%%
\Rule{
  \X \not\in \t' \and \Match{\t_1}{\t'}{\v}{\t_2}
}{
  \Match{\Exist\Xt{\t_1}}{\t'}{\v}{\Exist\Xt{\t_2}}
}  

\Rule{
  \Unify\X{\t'}{\v}{\t_2} \and \Match{\Subst{\t_1}\X{\t_2}}{\t'}{\v}{\t_3}
}{
  \Match{\Exist\Xt{\t_1}}{\t'}{\v}{\t_3}
}  
    
\Rule{\UnifyGround{\t'}{\v'}}{\Match\v{\t'}{\v'}\v}  

\\ %%%%%%%%%%%%%%%%%%%%%%%%%%%%%%%%%%%%%%%%%%%%%%%%%%%%%%%%%%%%%%%%%%%%%%%%%%%

\Rule{}{ \Unify\X\X\t\t }

\Rule{
  \exists i\in [1,n]. \Unify\X{\t_i}{\t_i'}\t
}{
  \Unify\X{\TyT{\t_1\dots\t_n}}{\TyT{\t_1'\dots\t_n'}}\t
}

\Rule{
  \exists i\in [1,n]. \Unify\X{\t_i}{\t_i'}\t
}{
  \Unify\X{\Tuple{\t_1\dots\t_n}}{\Tuple{\t_1'\dots\t_n'}}\t
}

\Rule{
  \exists i\in [1,n]. \Unify\X{\t_i}{\t_i'}\t
}{
  \Unify\X{\Union{\t_1\dots\t_n}}{\Union{\t_1'\dots\t_n'}}\t
}

\Rule{
  \X\not=\Y \and \exists i\in [1,2]. \Unify\X{\t_i}{\t_i'}\t
}{
  \Unify\X{\Exist{\Y^{\t_1}}{\t_2}}{\Exist{\Y^{\t_1'}}{\t_2'}}\t
}

\\

\Rule{}{
  \UnifyGround\X\t
}

\Rule{
    \UnifyGround{\t_1}{\t_1'} \and \UnifyGround{\t_n}{\t_n'}
}{
  \UnifyGround{\Exist{\X^{\t_1}}{\t_2}}{\Exist{\X^{\t_1'}}{\t_2'}}
}

\Rule{
    \UnifyGround{\t_1}{\t_1'} ~\dots~ \UnifyGround{\t_n}{\t_n'}
}{
  \UnifyGround{\Union{\t_1\dots\t_n}}{\Union{\t_1'\dots\t_n'}}
}

\Rule{
    \UnifyGround{\t_1}{\t_1'} ~\dots~ \UnifyGround{\t_n}{\t_n'}
}{
  \UnifyGround{\Tuple{\t_1\dots\t_n}}{\Tuple{\t_1'\dots\t_n'}}
}

\Rule{
    \UnifyGround{\t_1}{\t_1'} ~\dots~ \UnifyGround{\t_n}{\t_n'}
}{
  \UnifyGround{\TyT{\t_1\dots\t_n}}{\TyT{\t_1'\dots\t_n'}}
}

\end{mathpar}\caption{Auxiliary definitions}\label{aux}
\end{figure}

Rule \RSet extends the generator to sets of values by union of the
pointwise application of \Gen\cdot\cdot. As mentioned above, any given
\v can derive multiple sets of values, we assume that \Gen\v S returns
the union of all such sets.

Rule \RUnion generates the subtypes of a union as the union of the
pointwise sets for each element of the union. Unions are not part of
the set of a run-time values.

Rule \RTuple deals with tuples. First, subtypes are generated
pointwise for every element $\t_i$ of the tuple type. The result is
obtained as the combination of every possible tuple that could be
obtained from the values in each $S_i$, this combination is written
\Setify{\Tuple{S_1\dots S_2}}, note that we take liberty with the
syntax by creating a tuple of sets, the role \Setify\cdot is to shake
this out into a set of well-formed tuples.

Rule \RExists introduces a type variable \X in some type $\t'$,
written \Exist\Xt{\t'} where \X has an upper bound of \t.  The rule
start by generating the set $S'$ of subtypes of the bound \t. Then for
each indivudal value \v in $S'$, substitute that value for \X in $\t'$
and generates all subtypes of the result.  The notation
\Setify{\Subst{\t'}\X{S'}} denotes the set of types obtained by
replacing \X in $\t'$ with each of the values in $S'$.

Rule \RDeclare deals with subtypes of \TyT{\m{\t_1}}. It
non-deterministically picks one of the declarations with \T as a
supertype. \TyT{\m{\t_1}} is an instantiation of the parent type \T,
with no free variables. The rule identifies a direct subtype
$\l<:\TyT{\m{\t_2}}$ where \m{\t_2} is partial instantiation which may
have some free variables corresponding to some of the introductions
occurring in \l. To fix the mind consider the following example:

\newcommand{\Int}{\EM{\mathit{Int}}}

\newcommand{\Dict}{\EM{\mathit{Dict}}}
\newcommand{\IntDict}{\EM{\mathit{IntDict}}}


$\Int <: \Any$

$\Exist{\X^\Any}{\Exist{\Y^\Any}{\Ty\Dict{\X,\Y}}} <: \Any$

$\Exist{\X^\Any}{\Ty\IntDict{\X}} <: \Ty\Dict{\X, \Int}$

The query \Gen{\Ty\Dict{\Int,\Int}}S returns
$S=\Set{\Ty\Dict{\Int,\Int}, \Ty\IntDict{\Int}}$ as the only subtype
of \Ty\Dict{\Int,\Int} is \Ty\IntDict{\Int}.

On the other hand \Gen{\Ty\Dict{\Int,\Any}}S returns only the
singleton \Ty\Dict{\Int,\Any}. This is because parametric types are
invariant and thus $\Ty\Dict{\Int,\Int} \not<: \Ty\Dict{\Int,\Any}$.

To return to \RDeclare, if there is a candidate subtype,
\Match\l{\TyT{\m{\t_1}}}{\TyT{\m{\t_2}}}\t will strip the declaration
of variables that are bound and ensure that the parent instance
matches the one in the declaration.

In the auxiliaries, \Match\l\t\v{\t'} takes three arguments, the first
is a declaration with possibly some existentials, the second is the
instantiation of the parent type, the third is the declaration of the
parent type, and it returns a new type for \l with all the bound
variables stripped off. Match also ensures that \t and \v are similar.

Unify finds a binding for a variable.

$\cong$ ensures that two terms have the same structure module
variables occuring on the left hand side.

$\X\not\in\t$ indicates that variable \X does not occur free in type
\t, \Subst\t\X{\t'} denotes the substitution of $\t'$ for \X in type
\t.










\end{document}


\section{Type Language}

The type language supported here streamlines the syntax of Julia but should
be equivalent module a number of simplification that we discuss next.
\begin{itemize}
\item Any Julia type \c{t} can be translated to a type term $t$.
\item A Julia tuple, \c{Tuple\{t,t\}} is written \Tuple tt. Without
  loss of generality, we only support 2-tuples. Julia allows any
  arity. \c{NamedTuple}s are not supported currently.
\item A Julia union type \c{Union\{t,t\}} is written \Union
  tt. Without loss of generality we limit the presentation to
  2-unions.
\item A Julia abstract type or struct \c T is written $T()$, a Julia
  parametric type \c{Vector\{Int32\}} is written
  $\mathit{Vector}(\mathit{Int32})$. A parametric type has one or more
  parameters, written \m t.
\item The Julia type \c{Any} belongs to the set of type constructor
  denoted by $T$ and is written $\top$.
\item In Julia, the parametric type \c{Vector\{T\} where T} can be
  abbreviated \c{Vector}, we assume that all such abbreviations are
  reversed.
\item In Julia the notation \c{Foo\{<:Bar\}} is an abbreviation
  for \c{Foo\{X\} where X <: Bar} where \c X is assumed to be fresh.
  We assume that all such abbreviations are reversed.
\item Type parameters are non-recursive, \c{Foo\{ X <: Vec\{Y\}, Y <:
  Bar\{X\}\}} does not occur.
\item In Julia, types introduce by a \c{struct} are \emph{concrete},
  tuples are concrete. Other named types, including \c{Any}, are
  \emph{abstract}. We assume the presence of predicates to differentiate
  the two.
\item Values have types that are either $T(\m t)$ (possibly parametric
  types) where $T$ is concrete, or \Tuple tt. The inner types are not
  restricted.
\item In Julia, a Union-all type is written \c{T\{X\} where X <: t},
  they are written \Exist Xt{T(X)}. Julia allows lower bounds, we
  restrict the type language and ignore them. This is a loss of
  expressiveness.
\end{itemize}

\subsection{Types}

The syntax of types is defined as follows:



\newcommand{\NI}{\noindent}

\subsection{Examples}

\NI
The Julia type \c{Tuple\{Vector\{Vector\{T\} where T\}, ABM, AbstractArray\}}
can be shortened to (loosing an irrelevant type in the tupple to get a 2-tuple)

\[\text{
\c{Tuple\{Vec\{Vec\{T\} where T\}, Arr\}}
}\]

\NI
and written as (note that the type \c{AbstractArray} is a union-all abbreviation):

\newcommand{\tVec}{\EM{\mathit{Vec}}\xspace}
\newcommand{\tArr}{\EM{\mathit{Arr}}\xspace}

\[
\Tuple{{\tVec(\Exist X{\top}{\tVec(t)})}}{\Exist Y{\top}{\tArr(Y)}}
\]

\NI
The Julia type \c{Tuple\{DenseCuArray\{Complex\{<:cufftReals\}\},Int32\}} can be shortened to

\[\text{
\c{Tuple\{ D \{ C \( <:R \}\}, I \}}
}\]

\NI and written:

\[
\Tuple {\Exist XR{D(C(X))}}{I}
\]

\NI
The Julia type \c{Tuple\{NTuple\{N,Distribution\{<:ArrayLikeVariate\{M\}\}\} where N,Int32\} where M} can be shortened to:

\[\text{
\c{Tuple\{
          NT \{
               N,
               D \{ <: Arr\{ N \}\}
               \} where N,
          I   
          \} where M\}}
}\]

\NI and written (with parentheses for clarity):

\[
\Exist M{\top}{(\Tuple { \Exist N\top{ NT(N, \Exist X{\tArr(N)}{D(X)})}} {I})}
\]


\subsection{Definitions}

Datatype definitions are as follows:

\[\begin{array}{cl@{\quad}ll}
\end{array}\]

The goal of this note is to think about how to generate a subtype $v$
of some method signature $d$ given a set of datatypes \m d.


\section{Generation}






A{T<:Int}          <: B{S, 
A{T}               <: B{C{Int}, T}
A{T<:Int, S<:B{T}} <: B{T}

A{T, S}            <: B{Union{T, S}}

A{T, S}            <: B{Union{Ref{T}, S}}

A{T}               <: B           minimal recursive case !



\end{document}
